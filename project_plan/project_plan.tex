\documentclass[a4paper,11pt]{report}
\usepackage[utf8]{inputenc}\usepackage{graphicx}
\graphicspath{ {.} }

% Title Page
\title{
  {Pattern Generation in Natural Systems}\\
  {\large Project Plan}
}
\author{
  {by Anna Ruth Rowan}\\
  {supervised by Richard Clayton}\\
  \\
  {COM3600}
}

\begin{document}
\maketitle

\section*{Introduction}
% background, clear description of project, cite any literature read to date
The subject I would like to address in this dissertation is natural pattern generation - specifically those involving activator-inhibitor (or A-I) systems. This covers a wide range of processes undertaken by living organisms as well as happening on a larger scale such as in dune and star formation (as mentioned by Meinhardt in 'The Algorithmic Beauty of Sea Shells') but the focus will likely be on pattern generation in living organisms. 

Seashells make an excellent starting point for exploring the nature of activator-inhibitor systems since the area of growth is often nearly one-dimensional. This results in a visual window into the mechanics behind the pattern generation since one of the spacial dimensions effectively shows time. Another aspect that makes sea shells interesting is their startling resemblance to one-dimensional cellular automata, in which completely chaotic patterns can arise from a basic set of rules. 
While discussing the appearance of Fibonacci numbers in the spirals of a range of unrelated plant species, and the A-I system that lead to their appearance, during her 2012 talk titled 'Spirals, Fibonacci, and Being a Plant' Victoria Hart said:
``These incredibly intricate patterns can result from utterly simple beginnings''
which suggests that other, far more complex organisms might also be modelled with relatively simple rules. In fact, both Meinhardt in the previously mentioned book and Turing in his paper 'The Chemical Basis of Morphogenesis' discuss models for the growth of embryos based on an activator-inhibitor system.

What I intend to do in this project is to model of the pattern generation in various species of sea shell, implement the model(s) in a program, and explore how these models might be expanded and adapted to apply to other natural processes such as the morphogenesis of embryos.

\section*{Analysis}
% discussion of problems to be solved and possible techniques and tools
There are three basic sections to this project: designing the model for generating sea shell patterns, implementing that model, and then the expansion and exploration of more complex systems. The first part can be achieved through the use of partial differential equations like the kind that Meinhardt used in his models. For the second part a programming language with a good deal of support for maths would be useful, so Python would be a likely choice. Matlab would also work, but I am more familiar with the former. Finally the last part is the least concrete and so will require more research to plan in detail. I hope to either find one or two relatively complicated systems to focus on or perhaps look at a larger number of simple subjects for exploration such as the morphogenesis of plants.

\section*{Plan of Action}
\includegraphics[width=\textwidth,height=\textheight,keepaspectratio]{plan_of_action.png}
The first thing that need will need doing is to read up on the subjects of cellular automata, X and morphogenesis. During this phase a decision should also be made on what the exact focus of the latter part of project will be.

After the research is begun I can start on the literature review.

Following that the work on the model and modelling program can begin. Realistically the writing of the dissertation itself wont begin until I'm well into writing the model, so that should be done as soon as possible.

\end{document}          

%