\documentclass[a4paper,10pt]{scrreprt}
\usepackage[utf8]{inputenc}
\usepackage{listings}

% Title Page
\title{Pattern Generation in Natural Systems}
\author{Anna Ruth Rowan}


\begin{document}
% Title, name, supervisor, module code, date, and the following statement:     
% "This report is submitted in partial fulfilment of the requirement for the degree of [Degree Title] by [Full Name]".    
% The title the dissertation ends up with need not be the one it started with in the project choice stage more than a year earlier but it should be meaningful.  "My Design Project" is not meaningful.
\maketitle

%The second page should be the following signed declaration: 
%"All sentences or passages quoted in this report from other people's work have been specifically acknowledged by clear cross-referencing to author, work and page(s). Any illustrations which are not the work of the author of this report have been used with the explicit permission of the originator and are specifically acknowledged. I understand that failure to do this amounts to plagiarism and will be considered grounds for failure in this project and the degree examination as a whole.
% Name: 
% Signature: 
% Date:"
%The dissertation will not be accepted without this signed declaration. 

\begin{abstract}
% Abstract: briefly go over the background, project aims and main achievements of the project
% R: I would recommend doing this at the very end, so leave for now. 
\end{abstract}

%Acknowledgements: Thanks to whoever may have helped you in any way - both serious and a bit of fun.

% Contents: Includes titles and page numbers of all sections and subsections. Chapter 1 begins on page 1. Use Roman numerals for all previous pages, e.g.. title page (i), signed declaration (ii) abstract (iii), acknowledgements (iv) and contents (v-?). 
% It is often best include a separate list of all the figures in the dissertation (figure number, label, page number), and a separate list of all tables in the dissertation (table number, label, page number).

% Introduction: the background, aims, objectives and constraints in more detail and summarise the remaining chapters
% - the aim is to find a model that balances between achieving good diversity and remaining as simple as possible.
% - one constraint is the ability to enumerate pattern diversity.

% R: Think about the introduction as a funnel that starts wide and narrows down to your aim (patterns are widespread in natural systems, models provide a way to understand how patterns form, the aim of this project was to implement  a subset of these models and examine some of they properties).

% Make sure your aim is carefully written and focussed -- remember this is a dissertation project, not a 10 year research programme :-)

\chapter{Introduction}
This is chapter 1!
% Lit Review: review related literature, draw parallels between this work and theirs.
% - still not sure what to do about this bit.

% R: In the Lit Review you should describe other work that has been done (Turing, Meinhardt etc), including the equations that have been used for modelling patterns, and explain how your work extends this -- i.e. you are going to take the equations from the Meinhardt paper, implement them in Python, and examine the range of patterns that can be obtained as the model parameters vary.

\chapter{Lit Review (maybe rename?)}

chapter 7, p104 talks about complexity
% Requirements and Analysis: state objectives and break into steps. Discuss possible methods other than what was implemented. Discuss testing and evaluation.
% - not sure what this means in our case, do they want us to discuss the algorithm at this point or is that design?
% - probably discuss the method in the previous email for testing and evaluation.

% R: Based on your analysis in the Lit Review, describe the equations you need to solve, how you plan to solve them numerically (i.e. by approximation), what the limitations are (i.e. the need to meed the stability criterion), how you plan to display the results. Also, how you plan to do your experiments where the model parameters are varied, and how you classify and quantify the model outputs.

\chapter{Requirements and Analysis}



% Design: explain and justify design techniques, discuss trade offs.
% - what do they mean by design technique?
% - talk about the algorithm now?
%Implementation and Testing:
% - I'm really not sure what the difference is between this section and the last.

% R: I think you can merge these two sections into one chapter, covering the design, implementation and testing of your Python code.

\chapter{Design, Implementation and Testing}

In order to properly explore the range of possible outcomes for each equation we required the program to take a list of values for each parameter and then run the model repeatedly - cycling through every combination of the given parameter ranges. 
% Results and Discussion: present and discuss findings, look at which goals were achieved and which weren't, discuss further explorations.

% R: You should begin to think about what graphs and figures are going to go in this chapter, and you could even start to put in placeholders for these.

% The discussion is almost an inverse of the introduction -- staring with your main results, how do these complement and extend previous work, what is the relevance for natural systems (i.e. do these models shed light on the actual mechanisms?).

\chapter{Results and Discussion}
% Conclusions: bring together previous points.

% R: Should be short -- summarising what you have achieved.

\chapter{Conclusions}
% References
\chapter{Appendices}
\section{Appendix A}
This is the Python generator function used to iterate over every combination of a set of a set of parameters.
\begin{lstlisting}
 def product(param_list, param_set):
    for val in param_list[0][1]:
        # set the value of the first parameter
        param_set[param_list[0][0]] = val
        # if this isn't the last parameter, recurse
        if len(param_list) is not 1:
            for prod in product(param_list[1:], param_set):
                yield prod
        # else we have a complete parameter set and can yield it
        else:
            yield param_set
\end{lstlisting}


\end{document}          
